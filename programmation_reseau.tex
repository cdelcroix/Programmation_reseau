%%%%%%%%%%%%%%%%%%%%%%%%%%%%%%%%%%%%%%%%%%%%%%%%%%%%%%%%%%%%%%%%%%%%%%
% LaTeX Example: Project Report
%
% Source: http://www.howtotex.com
%
% Feel free to distribute this example, but please keep the referral
% to howtotex.com
% Date: March 2011 
% 
%%%%%%%%%%%%%%%%%%%%%%%%%%%%%%%%%%%%%%%%%%%%%%%%%%%%%%%%%%%%%%%%%%%%%%

%couleur dans code avec verbatim ?


% Edit the title below to update the display in My Documents
%\title{Project Report}
%
%%% Preamble
\documentclass[paper=a4, fontsize=12pt]{article}
\usepackage[T1]{fontenc}
\usepackage{fourier}
\usepackage[utf8]{inputenc}
\usepackage[french]{babel}
\usepackage{listings}


% English language/hyphenation
\usepackage[protrusion=true,expansion=true]{microtype}	
\usepackage{amsmath,amsfonts,amsthm} % Math packages
\usepackage[pdftex]{graphicx}	
\usepackage{url}
\usepackage[bottom=10em]{geometry}
\usepackage{float}
\usepackage{xcolor}
\usepackage{enumitem}
\renewcommand\descriptionlabel[1]{\textbf{#1 :}}
\usepackage{pdfpages}
\usepackage{rotating}

%%% Custom sectioning
%\usepackage{sectsty}
%\allsectionsfont{\normalfont\scshape}

%% Language definition package (for XML Annexe)
\usepackage{listings}
\usepackage{color}

%% Local modification of margins
\newenvironment{changemargin}[2]{\begin{list}{}{%
      \setlength{\topsep}{0pt}%
      \setlength{\leftmargin}{0pt}%
      \setlength{\rightmargin}{0pt}%
      \setlength{\listparindent}{\parindent}%
      \setlength{\itemindent}{\parindent}%
      \setlength{\parsep}{0pt plus 1pt}%
      \addtolength{\leftmargin}{#1}%
      \addtolength{\rightmargin}{#2}%
    }\item }{\end{list}}
%%

%%% Custom headers/footers (fancyhdr package)
%\usepackage{fancyhdr}
%\pagestyle{fancyplain}
%\fancyhead{}											% No page header
%\fancyfoot[L]{}											% Empty 
%\fancyfoot[C]{}											% Empty
%\fancyfoot[C]{\thepage}									% Pagenumbering
%\renewcommand{\headrulewidth}{0pt}			% Remove header underlines
%\renewcommand{\footrulewidth}{0pt}				% Remove footer underlines
%\setlength{\headheight}{13.6pt}


%%% Equation and float numbering
\numberwithin{equation}{section}		% Equationnumbering: section.eq#
\numberwithin{figure}{section}			% Figurenumbering: section.fig#
\numberwithin{table}{section}				% Tablenumbering: section.tab#

%Graphics path
%\graphicspath{./Images/}

%%% Maketitle metadata
\newcommand{\horrule}[1]{\rule{\linewidth}{#1}} 	% Horizontal rule

\title{
\begin{figure}[H]
\centering
\includegraphics[]{Illustrations/eirb.jpg}
\end{figure}
  %\vspace{-1in} 			
  \usefont{OT1}{bch}{b}{n}
  \horrule{1.5pt} \\[0.5cm]	
  \Huge \textbf{Rapport final} \\ [10pt]
  \Huge Programmation réseau par sockets \\ [15pt]
  \LARGE Année scolaire 2015-2016 \\ 
  \horrule{1.5pt} \\[0.5cm]
  %
}

\author{
  \huge \underline{Professeur} : \LARGE Patrice KADIONIK\\[20pt]
  \normalfont 							
  \huge \textbf{Binôme} :\\[10pt]
\Large Adrian BARBE - barbe.adrian@hotmail.fr\\[5pt]
 \Large Charles DELCROIX - cdelcroix@enseirb-matmeca.fr \\[5pt]
%rajouter adresses emails
  \normalsize
}
\date{}

%%% Begin document
\begin{document}

\maketitle
\newpage
 \Large Appréciation du professeur:\\
 \vspace{10\baselineskip}
  \horrule{1.5pt} \\[0.5cm]
\horrule{1.5pt} \\[0.5cm]
\normalsize
\tableofcontents

\newpage
%-------------------------------------------------------------------------------------------------------------------------------------
\section{But des travaux pratiques}
Le but de ces TP est de maîtriser la programmation réseau par sockets en langage C sous UNIX.
%-------------------------------------------------------------------------------------------------------------------------------------
\section{Utilisation des commandes unix}
\subsection{Utilisation des commandes d'analyse réseau}
Le but de netstat est de nous informer de tous les services qui tournent sur la machine.\\
Les services sont alors divises en trois paquets: NTP, socket du domaine UNIX et TCP.\\
netstat-a affiche quand à lui la table de routage IP du noyau
\\telnet a pour but de se connecter à distance à une machine (nous l'utilisons pour le test de serveurs).
\subsection{Etude de quelques services Internet}
telnet www.enseirb.fr 80 puis en appuyant sur entrée cela nous fait "sauter une ligne". En appuyant sur espace puis entrée cela nous affiche le code html de la page visitée puis cela nous ferme la connection.
\\Après avoir récupéré index.html, le code de retour est "200". Ce code nous indique qu'il n'y a pas eu d'erreur.
%-------------------------------------------------------------------------------------------------------------------------------------
\section{Séances 1, 2 et 3: Programmation réseau par sockets}
\subsection{Analyse d'un programme client TCP}

On utilise une socket de type TCP. En effet on remarque la présence de SOCK\_STREAM dans le code. 
Pour un client TCP la procédure standard est:

\begin{center}
socket() \\
connect()\\
write/read()\\
\end{center}
On retrouve l'enchaînement suivant dans notre fichier.
Le programme nous renvoie un code d'erreur 331, on demande d'établir une connection, il nous demande un mot de passe.

\subsection{Client TCP ftp myftp}
Ajout effectué sur le code:
\begin{verbatim}
while(1){

n=read(0, buf, sizeof(buf));
write(sd, buf, n);
n=read(sd, buf, sizeof(buf));
write(1, buf, n);
   
if(strncmp(buf,"QUIT\n",4)==0){
     close(sd);
     exit(0);
}
}
\end{verbatim}
\begin{figure}[h!]
\centerline{\includegraphics[width=10cm]{Illustrations/q4}}
\caption{\label{Illustrations/q4} Nouvelle entrée}
\end{figure}
\subsection{Client TCP mydateTCP}
Numéro de port: 13. Protocole de transport internet à utiliser: TCP.\\
Par rapport au programme \textit{myftp.c} on change donc le numéro de port de 21 à 13.
\begin{figure}[h!]
\leftline{\includegraphics[width=12cm]{Illustrations/q5}}
\caption{\label{Illustrations/q5}Retour de la date}
\end{figure}
\subsection{Client UDP mydateUDP}
Le but de ce fichier et de renvoyer la date et l'heure gràce à un client UDP (mode non connecté). Pour ce faire il faut modifier le fichier \textit{myftp.c}.
\\
Il faut remplacer SOCK\_ STREAM par SOCK\_ DGRAM (propre à l'UDP) lors de la création de la socket.
\begin{verbatim}
/* Creation de la socket UDP */
if((sd = socket(AF_INET, SOCK_DGRAM, 0)) < 0) {
	printf("Probleme lors de la creation de socket\n");
	exit(1);
}
\end{verbatim}
Pour mettre le service en oeuvre (\textcolor{green}{FTPPORT 13}) on écrit alors le code suivant:

\begin{verbatim}
/* Ecriture sur stdout */
//on envoie la requête
sendto(sd,buf,sizeof(buf),0,(struct sockaddr*)&sa,sizeof(sa));
i=sizeof(sa);
n =recvfrom(sd, buf,sizeof(buf),0,(struct sockaddr*)&sa,&i);
//on affiche dans le buffer ce qu'on reçoit
write(1,buf,n);
\end{verbatim}

\subsection{Analyse d'un programme serveur TCP}
Socket utilisé: TCP.\\

Pour un serveur TCP la procédure standard est:
\begin{center}
socket()\\
bind()\\
listen()\\
accept()\\
read/write()\\
\end{center}
On retrouve l'enchaînement suivant dans notre fichier.
\\Ce programme compte le nombre de connexion.\\
On le lance grâce à la commande ./pingserveur0 22222\\
Puis on l'appelle grâce à telnet localhost 22222
\begin{figure}[h!]
\centerline{\includegraphics[width=10cm]{Illustrations/q7}}
\caption{\label{Illustrations/q7} Côté serveur}
\end{figure}
\begin{figure}[h!]
\centerline{\includegraphics[width=10cm]{Illustrations/q7bis}}
\caption{\label{Illustrations/q7bis} Nouvelle connection}
\end{figure}
\subsection{Serveur TCP pingserveurTCP}
\begin{verbatim}
while(1){
int n=read(newsd,buf,sizeof(buf));
write(1,buf,n);
write(newsd,buf,n);
}
\end{verbatim}
On ne peut pas prendre n'importe quelle valeur de port. En effet certains sont déjà attribués.

\begin{figure}[h!]
\leftline{\includegraphics[width=14cm]{Illustrations/q8}}
\caption{\label{Illustrations/q8}Côté serveur}
\end{figure}
\newpage%sinon ça bug...
\begin{figure}[h!]
\centerline{\includegraphics[width=10cm]{Illustrations/q8bis}}
\caption{\label{Illustrations/q8bis} Côté client}
\end{figure}
\subsection{Serveur UDP pingserveurUDP, client TCP pingclientUDP}
On souhaite réaliser un écho/ping entre un serveur et un client, ç'est à dire que le serveur renvoie tout ce qu'il a reçu de la part du client. 
\\Après avoir copier le fichier \textit{pingserveurTCP0.c} dans les fichiers  \textit{pingserveurUDP.c} et  \textit{pingclientUDP.c}, on modifie le clientUDP:
\begin{verbatim}
while(1){
//On envoie ce qu'on lit sur l'entrée standard
m=read(0,buf,sizeof(buf));
sendto(sd,buf,m,0,(struct sockaddr*)&sa,sizeof(sa));

//On affiche la réponse du serveur
i=sizeof(sa);
n =recvfrom(sd, buf,sizeof(buf),0,(struct sockaddr*)&sa,&i);
write(1,buf,n);
}
\end{verbatim}
Ainsi que le serveurUDP de cette façon:
\begin{verbatim}
while(1) {
//On reçoit le message du client
salength = sizeof(sa);
int n=recvfrom(sd,buf,sizeof(buf),0,(struct sockaddr*)&sa,&salength);

//On le renvoit et on l'affiche (pas obligatoire)
write(1,buf,n);
sendto(sd,buf,n,0,(struct sockaddr*)&sa,sizeof(sa));

}	
\end{verbatim}
\newpage%encore le bug
\subsection{Serveur wwwserveur}
Pour accéder à un serveur web on utilise le protocole HTTP.
\begin{verbatim}
while(1) {
 /* Ne pas oublier : newsalength contient la taille de la structure sa attendue */
newsalength = sizeof(newsa) ;
if((newsd = accept(sd, ( struct sockaddr * ) &newsa, &newsalength)) < 0 ) {
printf("Erreur sur accept\n");
exit(1);
}
while(1){
write(newsd,accueil,sizeof(accueil));
close(newsd);
printf("Fin du serveur. Bye...\n");
exit(0);
}	
}
\end{verbatim}
\begin{figure}[h!]
\centerline{\includegraphics[width=15cm]{Illustrations/q10}}
\caption{\label{Illustrations/q10} Côté serveur}
\end{figure}
\begin{figure}[h!]
\centerline{\includegraphics[width=15cm]{Illustrations/q10bis}}
\caption{\label{Illustrations/q10bis} Côté client}
\end{figure}
\subsection{Serveur TCP lotoserveur}
Après le exit(1) du dernier programme on remplace la boucle while(1) par:
\begin{verbatim}
for(i=0;i<6;i++){
j=((rand())%49)+1;
sprintf(buf,"%d",j);
write(newsd,buf,sizeof(buf));
}
\end{verbatim}
%%% End document
\end{document}
