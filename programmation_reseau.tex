%%%%%%%%%%%%%%%%%%%%%%%%%%%%%%%%%%%%%%%%%%%%%%%%%%%%%%%%%%%%%%%%%%%%%%
% LaTeX Example: Project Report
%
% Source: http://www.howtotex.com
%
% Feel free to distribute this example, but please keep the referral
% to howtotex.com
% Date: March 2011 
% 
%%%%%%%%%%%%%%%%%%%%%%%%%%%%%%%%%%%%%%%%%%%%%%%%%%%%%%%%%%%%%%%%%%%%%%

% Edit the title below to update the display in My Documents
%\title{Project Report}
%
%%% Preamble
\documentclass[paper=a4, fontsize=12pt]{article}
\usepackage[T1]{fontenc}
\usepackage{fourier}
\usepackage[utf8]{inputenc}
\usepackage[french]{babel}

% English language/hyphenation
\usepackage[protrusion=true,expansion=true]{microtype}	
\usepackage{amsmath,amsfonts,amsthm} % Math packages
\usepackage[pdftex]{graphicx}	
\usepackage{url}
\usepackage[bottom=10em]{geometry}
\usepackage{float}
\usepackage{xcolor}
\usepackage{enumitem}
\renewcommand\descriptionlabel[1]{\textbf{#1 :}}
\usepackage{pdfpages}
\usepackage{rotating}

%%% Custom sectioning
%\usepackage{sectsty}
%\allsectionsfont{\normalfont\scshape}

%% Language definition package (for XML Annexe)
\usepackage{listings}
\usepackage{color}

%% Local modification of margins
\newenvironment{changemargin}[2]{\begin{list}{}{%
      \setlength{\topsep}{0pt}%
      \setlength{\leftmargin}{0pt}%
      \setlength{\rightmargin}{0pt}%
      \setlength{\listparindent}{\parindent}%
      \setlength{\itemindent}{\parindent}%
      \setlength{\parsep}{0pt plus 1pt}%
      \addtolength{\leftmargin}{#1}%
      \addtolength{\rightmargin}{#2}%
    }\item }{\end{list}}
%%

%%% Custom headers/footers (fancyhdr package)
%\usepackage{fancyhdr}
%\pagestyle{fancyplain}
%\fancyhead{}											% No page header
%\fancyfoot[L]{}											% Empty 
%\fancyfoot[C]{}											% Empty
%\fancyfoot[C]{\thepage}									% Pagenumbering
%\renewcommand{\headrulewidth}{0pt}			% Remove header underlines
%\renewcommand{\footrulewidth}{0pt}				% Remove footer underlines
%\setlength{\headheight}{13.6pt}


%%% Equation and float numbering
\numberwithin{equation}{section}		% Equationnumbering: section.eq#
\numberwithin{figure}{section}			% Figurenumbering: section.fig#
\numberwithin{table}{section}				% Tablenumbering: section.tab#

%Graphics path
%\graphicspath{./Images/}

%%% Maketitle metadata
\newcommand{\horrule}[1]{\rule{\linewidth}{#1}} 	% Horizontal rule

\title{
  %\vspace{-1in} 			
  \usefont{OT1}{bch}{b}{n}
  \horrule{1.5pt} \\[0.5cm]	
  \Huge \textbf{Rapport final} \\ [10pt]
  \Huge PFA - De la 3D vers la 2D \\ [15pt]
  \LARGE Année scolaire 2014-2015 \\ 
  \horrule{1.5pt} \\[0.5cm]
  %
}

\author{
  \huge \underline{Client} : \LARGE BLANC Carole, DESBARATS Pascal\\ [10pt] 
  \huge \underline{Encadrant} : \LARGE LOMBARDY Sylvain\\[20pt]
  \normalfont 							
  \huge \textbf{Equipe} : \Large BOHER Anaïs - CABON Yohann - CHAUVAT Magali \\[5pt]
  \Large LEVY Akané - MARCELIN Thomas \\[5pt]
  \Large MAUPEU Xavier - PHILIPPI Alexandre \\[10pt]		\normalsize
}
\date{}

%%% Begin document
\begin{document}
\maketitle
\newpage

\tableofcontents

\newpage

\section{But des travaux pratiques}
Le but de ces TP est de maîtriser la programmation réseau par sockets par sockets en langage C sous UNIX.
\section{Utilisation des commandes unix}
\subsection{Utilisation des commandes d'analyse réseau}
Le but de nestat est de nous informer de tous les services qui tournent sur la machine. Les services sont alors divises en trois paquets: NTP, socket du domaine UNIX et TCP.
\\netstat-a affiche la table de routage IP du noyau
\\telnet a pour but de se connecter à distance à une machine (nous utilisons pour le test de serveur).
\subsection{Etude de quelques services Internet}
telnet www.enseirb.fr 80 puis en appuyant sur entrée cela nous fait "sauter une ligne". En appuyant sur espace puis entrée cela nous affiche le code html de la page visitée puis cela nous ferme la connection.
\\Après avoir récupéré index.html, le code de retour est "200". Ce code nous indique qu'il n'y a pas eu d'erreur.
\section{Séances 1, 2 et 3: Programmation réseau par sockets}
\subsection{Analyse d'un programme client TCP}
\subsection{Client TCP ftp myftp}
\subsection{Client TCP mydateTCP}
Numéro de port: 13
\subsection{Client UDP mydateUDP}
\subsection{Analyse d'un programme serveur TCP}
\subsection{Serveur TCP pingserveurTCP}
\subsection{Serveur UDP pingserveurUDP, client TCP pingclientUDP}
\subsection{Serveur wwwserveur}
\subsection{Serveur TCP lotoserveur}

%%% End document
\end{document}
