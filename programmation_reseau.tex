%%%%%%%%%%%%%%%%%%%%%%%%%%%%%%%%%%%%%%%%%%%%%%%%%%%%%%%%%%%%%%%%%%%%%%
% LaTeX Example: Project Report
%
% Source: http://www.howtotex.com
%
% Feel free to distribute this example, but please keep the referral
% to howtotex.com
% Date: March 2011 
% 
%%%%%%%%%%%%%%%%%%%%%%%%%%%%%%%%%%%%%%%%%%%%%%%%%%%%%%%%%%%%%%%%%%%%%%

%couleur dans code avec verbatim ?


% Edit the title below to update the display in My Documents
%\title{Project Report}
%
%%% Preamble
\documentclass[paper=a4, fontsize=12pt]{article}
\usepackage[T1]{fontenc}
\usepackage{fourier}
\usepackage[utf8]{inputenc}
\usepackage[french]{babel}
\usepackage{listings}


% English language/hyphenation
\usepackage[protrusion=true,expansion=true]{microtype}	
\usepackage{amsmath,amsfonts,amsthm} % Math packages
\usepackage[pdftex]{graphicx}	
\usepackage{url}
\usepackage[bottom=10em]{geometry}
\usepackage{float}
\usepackage{xcolor}
\usepackage{enumitem}
\renewcommand\descriptionlabel[1]{\textbf{#1 :}}
\usepackage{pdfpages}
\usepackage{rotating}

%%% Custom sectioning
%\usepackage{sectsty}
%\allsectionsfont{\normalfont\scshape}

%% Language definition package (for XML Annexe)
\usepackage{listings}
\usepackage{color}

%% Local modification of margins
\newenvironment{changemargin}[2]{\begin{list}{}{%
      \setlength{\topsep}{0pt}%
      \setlength{\leftmargin}{0pt}%
      \setlength{\rightmargin}{0pt}%
      \setlength{\listparindent}{\parindent}%
      \setlength{\itemindent}{\parindent}%
      \setlength{\parsep}{0pt plus 1pt}%
      \addtolength{\leftmargin}{#1}%
      \addtolength{\rightmargin}{#2}%
    }\item }{\end{list}}
%%

%%% Custom headers/footers (fancyhdr package)
%\usepackage{fancyhdr}
%\pagestyle{fancyplain}
%\fancyhead{}											% No page header
%\fancyfoot[L]{}											% Empty 
%\fancyfoot[C]{}											% Empty
%\fancyfoot[C]{\thepage}									% Pagenumbering
%\renewcommand{\headrulewidth}{0pt}			% Remove header underlines
%\renewcommand{\footrulewidth}{0pt}				% Remove footer underlines
%\setlength{\headheight}{13.6pt}


%%% Equation and float numbering
\numberwithin{equation}{section}		% Equationnumbering: section.eq#
\numberwithin{figure}{section}			% Figurenumbering: section.fig#
\numberwithin{table}{section}				% Tablenumbering: section.tab#

%Graphics path
%\graphicspath{./Images/}

%%% Maketitle metadata
\newcommand{\horrule}[1]{\rule{\linewidth}{#1}} 	% Horizontal rule

\title{
  %\vspace{-1in} 			
  \usefont{OT1}{bch}{b}{n}
  \horrule{1.5pt} \\[0.5cm]	
  \Huge \textbf{Rapport final} \\ [10pt]
  \Huge Programmation réseau par sockets \\ [15pt]
  \LARGE Année scolaire 2015-2016 \\ 
  \horrule{1.5pt} \\[0.5cm]
  %
}

\author{
  \huge \underline{Encadrant} : \LARGE Patrice KADIONIK\\[20pt]
  \normalfont 							
  \huge \textbf{Equipe} : \Large Adrian BARBE - Charles DELCROIX \\[5pt]
%rajouter adresses emails
  \normalsize
}
\date{}

%%% Begin document
\begin{document}
\maketitle
\newpage

\tableofcontents

\newpage
%-------------------------------------------------------------------------------------------------------------------------------------
\section{But des travaux pratiques}
Le but de ces TP est de maîtriser la programmation réseau par sockets par sockets en langage C sous UNIX.
%-------------------------------------------------------------------------------------------------------------------------------------
\section{Utilisation des commandes unix}
\subsection{Utilisation des commandes d'analyse réseau}
Le but de netstat est de nous informer de tous les services qui tournent sur la machine.\\
Les services sont alors divises en trois paquets:
\begin{itemize}[label=$\square$,leftmargin=* ,parsep=0cm,itemsep=0cm,topsep=0cm]
\item NTP
\item socket du domaine UNIX 
\item TCP
\end{itemize}
netstat-a affiche quand à lui la table de routage IP du noyau
\\telnet a pour but de se connecter à distance à une machine (nous utilisons pour le test de serveur).
\subsection{Etude de quelques services Internet}
telnet www.enseirb.fr 80 puis en appuyant sur entrée cela nous fait "sauter une ligne". En appuyant sur espace puis entrée cela nous affiche le code html de la page visitée puis cela nous ferme la connection.
\\Après avoir récupéré index.html, le code de retour est "200". Ce code nous indique qu'il n'y a pas eu d'erreur.
%-------------------------------------------------------------------------------------------------------------------------------------
\section{Séances 1, 2 et 3: Programmation réseau par sockets}
\subsection{Analyse d'un programme client TCP}
On utilise une socket de type TCP. En effet on remarque la présence de SOCK\_STREAM dans le code. 
Pour un client TCP la procédure standard est:\\
socket()\\
connect()\\
write/read()\\
On retrouve l'enchaînement suivant dans notre fichier.
But du programme: code d'erreur 331, on demande d'établir une connection, il nous demande un mot de passe.
% vérifier programme d'origine
% ajouter copie écran
\subsection{Client TCP ftp myftp}
\begin{verbatim}

#define FTPPORT 21		/* Numero du port du serveur ftp */


main(argc,argv)
int argc ;
char *argv[] ;

{
int sd;
struct sockaddr_in sa;		/* Structure Internet sockaddr_in */
struct hostent *hptr ; 		/* Infos sur le serveur */

char *serveur ;        		/* Nom du serveur distant */
char c;

char msg[] = "USER TOTO\n"; 	/* Commande FTP */
char buf[256]; 			/* Buffer */
int n, i;


/* verification du nombre d'arguments de la ligne de commande */
if (argc != 2) {
	printf("myftp. Erreur d'arguments\n");
	printf("Syntaxe : %% myftp nom_serveur_ftp\n");
	exit(1);
}

/* Recuperation nom du serveur */
serveur = argv[1];

/* Recuperation des infos sur le serveur dans /etc/hosts pour par DNS */
if((hptr = gethostbyname(serveur)) == NULL) {
	printf("Probleme de recuperation d'infos sur le serveur\n");
	exit(1);
}

/* Initialisation la structure sockaddr sa avec les infos  formattees : */
/* bcopy(void *source, void *destination, size_t taille); 		*/
bcopy((char *)hptr->h_addr, (char*)&sa.sin_addr, hptr->h_length);

/* Famille d'adresse : AF_INET = PF_INET */
sa.sin_family = AF_INET;


/* Initialisation du numero du port */
sa.sin_port = htons(FTPPORT);

/* Creation de la socket TCP */
if((sd = socket(AF_INET, SOCK_STREAM, 0)) < 0) {
	printf("Probleme lors de la creation de socket\n");
	exit(1);
}

/* Etablissement de la connexion avec le serveur ftp */
if((connect(sd, (struct sockaddr *) &sa, sizeof(sa))) < 0 ) {
	printf("Probleme de connexion avec le serveur\n");
	exit(1);
}


/* Lecture de la banniere d'accueil du serveur ftp */
n = read(sd, buf, sizeof(buf));

/* Ecriture sur stdout */
write(1, buf, n);

/* Envoi de la commande ftp vers serveur ftp */
write(sd, msg, sizeof(msg));

/* Lecture de la reponse du serveur ftp */
n = read(sd, buf, sizeof(buf));
write(1, buf, n);


 while(1){

  n= read(0, buf, sizeof(buf));
   write(sd, buf, n);
  n=read(sd, buf, sizeof(buf));
   write(1, buf, n);
   
if(strncmp(buf,"QUIT\n",4)==0){
     close(sd);
     exit(0);
}

}

/* Fermeture de la socket */
close(sd);

exit(0);
}
\end{verbatim}
\begin{figure}[h!]
\centerline{\includegraphics[width=10cm]{Illustrations/q4}}
\caption{\label{Illustrations/q4} Nouvelle entrée}
\end{figure}
\subsection{Client TCP mydateTCP}
Numéro de port: 13 Protocole de transport internet: TCP
(pour lancer le programme ./mydateTCP brahmane-enseirb.fr)
%copie écran .
-> On change le port 13 -> 21
\subsection{Client UDP mydateUDP}
Le but de ce fichier et de renvoyer la date et l'heure gràce à un client UDP (mode non connecté). Pour ce faire il faut modifier le fichier \textit{myftp.c}.
\\
Il faut remplacer SOCK\_ STREAM par SOCK\_ DGRAM (propre à l'UDP) lors de la création de la socket.
\begin{figure}[h!]
\leftline{\includegraphics[width=12cm]{Illustrations/0}}
\caption{\label{Illustrations/0}Création de la socket UDP}
\end{figure}

Pour mettre le service en oeuvre (\textcolor{green}{FTPPORT 13}) on écrit alors le code suivant.

\begin{figure}[h!]
\leftline{\includegraphics[width=14cm]{Illustrations/1}}
\caption{\label{Illustrations/decoupage_etapes_2}Ecriture sur sdtout en UDP}
\end{figure}
On modifie SOCK\_STEAM par SOCK\_DGRAM
Programme réalisé:
\begin{verbatim}
sendto(sd,buf,sizeof(buf),0,(struct sockaddr*)&sa,sizeof(sa));
i=sizeof(sa);
n=recvfrom(sd,buf,sizeof(buf),0,(struct sockaddr*)&sa,&i);
write(1,buf,n);
\end{verbatim}
\subsection{Analyse d'un programme serveur TCP}
Socket utilisé: TCP
socket()
bind()
listen()
accept()
read/write()
\\On retrouve l'enchaînement classique des appels systèmes.
\\Ce programme compte le nombre de connexion.
./pingserveur0 22222
telnet localhost 22222
\begin{figure}[h!]
\centerline{\includegraphics[width=10cm]{Illustrations/q7}}
\caption{\label{Illustrations/q7} Côté serveur}
\end{figure}
\begin{figure}[h!]
\centerline{\includegraphics[width=10cm]{Illustrations/q7bis}}
\caption{\label{Illustrations/q7bis} Nouvelle connection}
\end{figure}
\subsection{Serveur TCP pingserveurTCP}
\begin{figure}[h!]
\centerline{\includegraphics[width=10cm]{Illustrations/q8}}
\caption{\label{Illustrations/q8} Côté serveur}
\end{figure}
\begin{figure}[h!]
\centerline{\includegraphics[width=10cm]{Illustrations/q8bis}}
\caption{\label{Illustrations/q8bis} Côté client}
\end{figure}
%un autre Q8bis ?
On ne peut pas prendre n'importe quel valeur de port. EN effet certain sont déjà attribués.
newsd->client
\begin{verbatim}
while(1){
int n=read(newsd,buf,sizeof(buf));
write(1,buf,n);
write(newsd,buf,n);
}
\end{verbatim}
Les valeurs des ports en cours d'utilisation ne peuvent être prises.
\subsection{Serveur UDP pingserveurUDP, client TCP pingclientUDP}
On souhaite réaliser un écho/ping entre un serveur et un client, ç'est à dire que le serveur renvoie tout ce qu'il a reçu de la part du client. 
\\Après avoir copier le fichier \textit{pingserveurTCP0.c} dans les fichiers  \textit{pingserveurUDP.c} et  \textit{pingclientUDP.c}, on modifie le clientUDP.

\begin{figure}[h!]
\leftline{\includegraphics[width=14cm]{Illustrations/3}}
\caption{\label{Illustrations/decoupage_etapes_2}Ping client UDP}
\end{figure}
Ainsi que le serveurUDP de cette façon.
\begin{figure}[h!]
\leftline{\includegraphics[width=14cm]{Illustrations/2}}
\caption{\label{Illustrations/decoupage_etapes_2}Ping serveur UDP}
\end{figure}

\subsection{Serveur wwwserveur}
Pour accéder à un serveur web on utilise le protocole HTTP.
\begin{figure}[h!]
\centerline{\includegraphics[width=10cm]{Illustrations/q10}}
\caption{\label{Illustrations/q10} Côté serveur}
\end{figure}
\begin{figure}[h!]
\centerline{\includegraphics[width=10cm]{Illustrations/q10bis}}
\caption{\label{Illustrations/q10bis} Côté client}
\end{figure}
\subsection{Serveur TCP lotoserveur}

%%% End document
\end{document}
